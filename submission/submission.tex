\documentclass[10pt,a4paper]{scrartcl}
% \usepackage[left=2cm,right=2cm,top=2cm,bottom=2cm]{geometry}

%Language settings
\usepackage[utf8]{inputenc}
\usepackage[T1]{fontenc}
\usepackage[german]{babel}
\usepackage[babel, german=quotes]{csquotes}

%Fonts
\usepackage{lmodern}
\usepackage{mathrsfs}

%Packages
\usepackage{amsmath}
\usepackage{amsfonts}
\usepackage{amssymb}
\usepackage{xcolor}
\usepackage{graphicx}
\usepackage{caption}
\usepackage[linesnumbered,ruled]{algorithm2e}

%Biblatex
\usepackage{hyperref}
\usepackage[style=alphabetic]{biblatex}
\addbibresource{bibliography.bib}

%Title, author, etc.
\title{Support vector machines mit Kernel Trick}
\author{Jonas Klug, Hendrik Siek und Tim Schlottmann \\ TU Hamburg }
\date{\today} 

\begin{document}

    \maketitle

    \section{Abstract}

    \section{Motivation -- Tim}

    \section{Grundlagen -- Tim}
        Support Vector Maschinen sind ein binärer Klassifizierer. Datenmengen werden also in zwei Klassen eingeteilt. Zur Unterteilung wird hierfür eine Hyperebene benutzt.
        
        \subsection{Hyperebene}

        \subsection{Langrange Multiplikatoren}

    \section{Kernel Trick -- Jonas}

    \section{Praktische Umsetzung -- Hendrik}



\end{document}